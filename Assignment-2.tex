\documentclass[journal,12pt,twocolumn]{IEEEtran}

\usepackage{setspace}
\usepackage{gensymb}
\singlespacing
\usepackage[cmex10]{amsmath}

\usepackage{amsthm}

\usepackage{mathrsfs}
\usepackage{txfonts}
\usepackage{stfloats}
\usepackage{bm}
\usepackage{cite}
\usepackage{cases}
\usepackage{subfig}

\usepackage{longtable}
\usepackage{multirow}

\usepackage{enumitem}
\usepackage{mathtools}
\usepackage{steinmetz}
\usepackage{tikz}
\usepackage{circuitikz}
\usepackage{verbatim}
\usepackage{tfrupee}
\usepackage[breaklinks=true]{hyperref}
\usepackage{graphicx}
\usepackage{tkz-euclide}

\usetikzlibrary{calc,math}
\usepackage{listings}
    \usepackage{color}                                            %%
    \usepackage{array}                                            %%
    \usepackage{longtable}                                        %%
    \usepackage{calc}                                             %%
    \usepackage{multirow}                                         %%
    \usepackage{hhline}                                           %%
    \usepackage{ifthen}                                           %%
    \usepackage{lscape}     
\usepackage{multicol}
\usepackage{chngcntr}

\DeclareMathOperator*{\Res}{Res}

\renewcommand\thesection{\arabic{section}}
\renewcommand\thesubsection{\thesection.\arabic{subsection}}
\renewcommand\thesubsubsection{\thesubsection.\arabic{subsubsection}}

\renewcommand\thesectiondis{\arabic{section}}
\renewcommand\thesubsectiondis{\thesectiondis.\arabic{subsection}}
\renewcommand\thesubsubsectiondis{\thesubsectiondis.\arabic{subsubsection}}


\hyphenation{op-tical net-works semi-conduc-tor}
\def\inputGnumericTable{}                                 %%

\lstset{
%language=C,
frame=single, 
breaklines=true,
columns=fullflexible
}
\begin{document}

\newcommand{\BEQA}{\begin{eqnarray}}
\newcommand{\EEQA}{\end{eqnarray}}
\newcommand{\define}{\stackrel{\triangle}{=}}
\bibliographystyle{IEEEtran}
\raggedbottom
\setlength{\parindent}{0pt}
\providecommand{\mbf}{\mathbf}
\providecommand{\pr}[1]{\ensuremath{\Pr\left(#1\right)}}\newcommand*{\permcomb}[4][0mu]{{{}^{#3}\mkern#1#2_{#4}}}
\newcommand*{\perm}[1][-3mu]{\permcomb[#1]{P}}
\newcommand*{\comb}[1][-1mu]{\permcomb[#1]{C}}
\providecommand{\qfunc}[1]{\ensuremath{Q\left(#1\right)}}
\providecommand{\sbrak}[1]{\ensuremath{{}\left[#1\right]}}
\providecommand{\lsbrak}[1]{\ensuremath{{}\left[#1\right.}}
\providecommand{\rsbrak}[1]{\ensuremath{{}\left.#1\right]}}
\providecommand{\brak}[1]{\ensuremath{\left(#1\right)}}
\providecommand{\lbrak}[1]{\ensuremath{\left(#1\right.}}
\providecommand{\rbrak}[1]{\ensuremath{\left.#1\right)}}
\providecommand{\cbrak}[1]{\ensuremath{\left\{#1\right\}}}
\providecommand{\lcbrak}[1]{\ensuremath{\left\{#1\right.}}
\providecommand{\rcbrak}[1]{\ensuremath{\left.#1\right\}}}
\theoremstyle{remark}
\newtheorem{rem}{Remark}
\newcommand{\sgn}{\mathop{\mathrm{sgn}}}
\providecommand{\abs}[1]{\vert#1\vert}
\providecommand{\res}[1]{\Res\displaylimits_{#1}} 
\providecommand{\norm}[1]{\lVert#1\rVert}
%\providecommand{\norm}[1]{\lVert#1\rVert}
\providecommand{\mtx}[1]{\mathbf{#1}}
\providecommand{\mean}[1]{E[ #1 ]}
\providecommand{\fourier}{\overset{\mathcal{F}}{ \rightleftharpoons}}
%\providecommand{\hilbert}{\overset{\mathcal{H}}{ \rightleftharpoons}}
\providecommand{\system}{\overset{\mathcal{H}}{ \longleftrightarrow}}
	%\newcommand{\solution}[2]{\textbf{Solution:}{#1}}
\newcommand{\solution}{\noindent \textbf{Solution: }}
\newcommand{\cosec}{\,\text{cosec}\,}
\providecommand{\dec}[2]{\ensuremath{\overset{#1}{\underset{#2}{\gtrless}}}}
\newcommand{\myvec}[1]{\ensuremath{\begin{pmatrix}#1\end{pmatrix}}}
\newcommand{\mydet}[1]{\ensuremath{\begin{vmatrix}#1\end{vmatrix}}}
\numberwithin{equation}{subsection}
\makeatletter
\@addtoreset{figure}{problem}
\makeatother
\let\StandardTheFigure\thefigure
\let\vec\mathbf
\renewcommand{\thefigure}{\theproblem}
\def\putbox#1#2#3{\makebox[0in][l]{\makebox[#1][l]{}\raisebox{\baselineskip}[0in][0in]{\raisebox{#2}[0in][0in]{#3}}}}
     \def\rightbox#1{\makebox[0in][r]{#1}}
     \def\centbox#1{\makebox[0in]{#1}}
     \def\topbox#1{\raisebox{-\baselineskip}[0in][0in]{#1}}
     \def\midbox#1{\raisebox{-0.5\baselineskip}[0in][0in]{#1}}
\vspace{3cm}
\title{AI1103-Assignment-2}
\author{Vamsi Preetham Jumala\\CS20BTECH11058}
\maketitle
\newpage
\bigskip
\renewcommand{\thefigure}{\theenumi}
\renewcommand{\thetable}{\theenumi}


Download all python codes from
\begin{lstlisting}
https://github.com/VamsiPreetham-21/AI1103-Assignment-2/blog/main/Assignment2.py
\end{lstlisting}

Download all latex codes from
\begin{lstlisting}
https://github.com/VamsiPreetham-21/AI1103-Assignment-2/blog/main/Assignment2.tex
\end{lstlisting}

GATE 2012(EC), Q37 :\\
A fair coin is tossed till a head appeared for the first time. The probability that the number of tosses required is odd,\\

Solution:\\
Let $X_n$ define a markov chain where n$\in$\brak{0,1,2,...}. O,E,F,S be four respective states denoting 

Here O,E states are transient and F,S states are absorbing. 
Let P denote the state transition matrix for the above markov chain.
\begin{align}P =
\begin{bmatrix}
0 & 0.5 & 0 & 0.5 \\
0.5 & 0 & 0.5 & 0 \\
0 & 0 & 1 & 0 \\
0 & 0 & 0 & 1
\end{bmatrix}
\end{align}
The standard form of the matrix is :
\begin{align}P = 
\begin{bmatrix}
I & O \\
R & Q 
\end{bmatrix}
\end{align}

where I,O are Identity and Zero matrices and R,Q are other sub-matrices respectively.After converting the transition matrix into the form of standard matrix we get
\begin{align} P = 
\begin{bmatrix}
1 & 0 & 0 & 0 \\
0 & 1 & 0 & 0 \\
0 & 0.5 & 0 & 0.5 \\
0.5 & 0 & 0.5 & 0 
\end{bmatrix}
\end{align}
The limiting matrix for absorbing markov chain is,
\begin{align}P = 
\begin{bmatrix}
I & O \\
FR & O
\end{bmatrix}
\end{align}

here F=\brak{I-Q}$^{-1}$ is called the fundamental matrix of P. Solving this leaves us with 
\begin{align}
P = 
\begin{bmatrix}
1 & 0 & 0 & 0 \\
0 & 1 & 0 & 0 \\
0.333 & 0.667 & 0 & 0 \\
0.667 & 0.333 & 0 & 0
\end{bmatrix}
\end{align}
Here an element $P_{ij}$ represents the probability of state j starting from state i. That means the probability of the first head to appear on odd numberth trail starting with a odd numberth trail is  
\begin{align}
   \Pr(A) = p_{14} 
   & = 0.667
\end{align}

\begin{table}
\centering
\begin{tabular}{|c|c|}
\hline
Symbol & State  \\ \hline
O                      & Odd try     \\ \hline
E                      & Even try    \\ \hline
F                      & Failure     \\ \hline
S                      & Success     \\ \hline
\end{tabular}
\caption{Symbols and States used in the Markov chain}
\label{tab:Table 5.10}
\end{table}

\centering
\includegraphics[scale=0.65]{}
\begin{figure}[h]
\caption*{\textbf{Markov chain diagram}}
\label{Eq.1.0.0}
\centering
\begin{tikzpicture}
    % Setup the style for the states
        \tikzset{node style/.style={state, 
                                    minimum width=1.2cm,
                                    line width=0.85mm,
                                    fill=gray!20!white}}
        % Draw the states
        \node[node style] at (0 , 0)      (bull)     {O};
        \node[node style] at (6 , 0)      (bear)     {E};
        \node[node style] at (0 , 6)     (stagnant) {S};
        \node[node style] at (6 , 6) 
        (loss) {F};
        % Connect the states with arrows
        \draw[every loop,
              auto=right,
              line width=0.7mm,
              >=latex,
              draw=red,
              fill=red]
            (stagnant)     edge[loop right]            node {1} (stagnant)
            (bull)     edge[bend right=20] node {$\dfrac{1}{2}$} (bear)
            (bear)     edge[bend right=20] node {$\dfrac{1}{2}$} (bull)
            (bull)     edge[bend left=20] node {$\dfrac{1}{2}$} (stagnant)
            (bear) edge[bend right=20]
            node{$\dfrac{1}{2}$} (loss)
            (loss) edge[loop right] node{1} (loss);
\end{tikzpicture}
\end{figure}



\end{document}
